\documentclass[12pt, titlepage]{article}
\usepackage{../../project_latex/report-style}
% \usepackage{../../project_latex/mishelper}

\usepackage{fullpage}
\usepackage[round]{natbib}
\usepackage{multirow}
\usepackage{booktabs}
\usepackage{tabularx}
\usepackage{graphicx}
\usepackage{float}
\usepackage{hyperref}
\usepackage[normalem]{ulem}
\hypersetup{
    colorlinks,
    citecolor=black,
    filecolor=black,
    linkcolor=red,
    urlcolor=blue
}
\usepackage[round]{natbib}

\newcounter{acnum}
\newcommand{\actheacnum}{AC\theacnum}
\newcommand{\acref}[1]{AC\ref{#1}}

\newcounter{ucnum}
\newcommand{\uctheucnum}{UC\theucnum}
\newcommand{\uref}[1]{UC\ref{#1}}

\newcounter{mnum}
\newcommand{\mthemnum}{M\themnum}
\newcommand{\mref}[1]{M\ref{#1}}

\title{SE 3XA3: Software Requirements Specification\\Title of Project}


\date{\today}

%\input{../../Comments}



%------------------------------------------------------------------------------
%------------------------------------------------------------------------------
% Document Info
%------------------------------------------------------------------------------
%------------------------------------------------------------------------------
\newcommand{\docTitle}{MIS}
\newcommand{\dueDate}{November 10, 2017}


%------------------------------------------------------------------------------
%------------------------------------------------------------------------------
% Revision Table
%------------------------------------------------------------------------------
%------------------------------------------------------------------------------
\newcommand{\revisionTable}{
	\begin{table}[hp]
		
		\begin{tabularx}{\textwidth}{p{3cm}p{2cm}X}
			\toprule {\bf Date} & {\bf Version} & {\bf Notes}\\
			\midrule
			
			10-Nov-2017 & 0.0 & Revision 0\\
			
			\bottomrule
		\end{tabularx}
		\caption{\bf Revision History}
	\end{table}
}




% Header
%-------------------------------------------------------------------------------
\usepackage{afterpage}
\usepackage{fancyhdr}
\pagestyle{fancy}
\setlength{\headheight}{15pt}
%\fancyhead[LE,RO]{\slshape \rightmark}
%\fancyhead[LO,RE]{\slshape \leftmark}
\fancyhead[LE,RO]{\slshape Assignment 1}
\fancyhead[LO,RE]{\slshape \leftmark}
\fancyfoot[C]{\thepage}
%\headrulewidth 0.4pt
%\footrulewidth 0 pt

\fancyhead{} % clear all header fields
%\fancyhead[RO,LE]{\textbf{\docTitle}}
\fancyhead[C]{\textbf{\docTitle}}
\fancyfoot{} % clear all footer fields
\fancyfoot[LE,RO]{\thepage}
\fancyfoot[LO,CE]{\dueDate}
%\fancyfoot[CO,RE]{\textbf{\teamname}}
\fancyfoot[CO,RE]{\teamname}
\renewcommand{\headrulewidth}{0.4pt}
%\renewcommand{\footrulewidth}{0.4pt}
\setlength{\headsep}{0.2in}




%%%%%%%%%%%%%%%%%%%%%%%%%%%%%%%%%%%%%%%%%%%%%%%%%%%%%%%%%%%%%%%%%%%%%%%%%%%%%%%%%%%%%
%%%%%%%%%%%%%%%%%%%%%%%%%%%%%%%%%%%%%%%%%%%%%%%%%%%%%%%%%%%%%%%%%%%%%%%%%%%%%%%%%%%%%
%	Macros
%%%%%%%%%%%%%%%%%%%%%%%%%%%%%%%%%%%%%%%%%%%%%%%%%%%%%%%%%%%%%%%%%%%%%%%%%%%%%%%%%%%%%
%%%%%%%%%%%%%%%%%%%%%%%%%%%%%%%%%%%%%%%%%%%%%%%%%%%%%%%%%%%%%%%%%%%%%%%%%%%%%%%%%%%%%


%%%%%%%%%%%%%%%%%%%%%%%%%%%%%%%%%%%%%%%%%%%%%%%%%%%%%%%%%%%%%%%%%%%%%%%%%%%%%%%%%%%%%
%
%  Create a module
%
%	\newModule{Name}
%		{Uses}
%		{ExportedTypes}{ExportedAccessPrograms}
%		{StateVariables}{StateInvariant}
%		{Assumptions}
%		{AccessPrograms}{LocalFuntions}
%
%%%%%%%%%%%%%%%%%%%%%%%%%%%%%%%%%%%%%%%%%%%%%%%%%%%%%%%%%%%%%%%%%%%%%%%%%%%%%%%%%%%%%%
\newcommand{\newModule}[9]{
	%\label{#1}
	\subsection* {Module}
		#1
	\subsection* {Uses}
		#2
	\subsection* {Syntax}
		\subsubsection* {Exported Types}
			#3
		\subsubsection* {Exported Access Programs}
			#4
	\subsection* {Semantics}
		\subsubsection* {State Variables}
			#5
		\subsubsection* {State Invariant}
			#6
		\subsubsection* {Assumptions}
			#7
		\subsubsection* {Access Routine Semantics}
			#8
		\subsubsection* {Local Functions}
			#9
}
\newcommand{\newTemplateModule}[9]{
	%\label{#1}
	\subsection* {Template Module}
		#1
	\subsection* {Uses}
		#2
	\subsection* {Syntax}
		\subsubsection* {Exported Types}
			#3
		\subsubsection* {Exported Access Programs}
			#4
	\subsection* {Semantics}
		\subsubsection* {State Variables}
			#5
		\subsubsection* {State Invariant}
			#6
		\subsubsection* {Assumptions}
			#7
		\subsubsection* {Access Routine Semantics}
			#8
		\subsubsection* {Local Functions}
			#9
}
\newcommand{\newConstantsModule}[6]{
	%\label{#1}
	\subsection* {Module}
		#1
	\subsection* {Uses}
		#2
	\subsection* {Syntax}
		\subsubsection* {Exported Constants}
			#3
		\subsubsection* {Exported Access Programs}
			#4
	\subsection* {Semantics}
		\subsubsection* {State Variables}
			#5
		\subsubsection* {State Invariant}
			#6
%		\subsubsection* {Assumptions}
%			#7
%		\subsubsection* {Access Routine Semantics}
%			#8
%		\subsubsection* {Local Functions}
%			#9
}

%%%%%%%%%%%%%%%%%%%%%%%%%%%%%%%%%%%%%%%%%%%%%%%%%%%%%%%%%%%%%%%%%%%%%%%%%%%%%%%%%%%%%%
%
% Create a new access program
%	\newAccessProgram{name}{output}{exception}
%	\newAccessProgram{someFunction(input1, input2)}{output equation}{SomeException}
%
%%%%%%%%%%%%%%%%%%%%%%%%%%%%%%%%%%%%%%%%%%%%%%%%%%%%%%%%%%%%%%%%%%%%%%%%%%%%%%%%%%%%%%
\newcommand{\newAccessProgram}[5]{
	\noindent #1:
		\begin{itemize}
		    \item input: #2
			\item transition: #3
			\item output: #4
			\item exception: #5
		\end{itemize}
}

\newcommand{\oldAccessProgram}[5]{
	\noindent \sout{#1}:
		\begin{itemize}
		    \item \sout{input: #2}
			\item \sout{transition: #3}
			\item \sout{output: #4}
			\item \sout{exception: #5}
		\end{itemize}
}

\newcommand{\revAccessProgram}[5]{
	\noindent \textcolor{red}{#1}:
		\begin{itemize}
		    \item \textcolor{red}{input: #2}
			\item \textcolor{red}{transition: #3}
			\item \textcolor{red}{output: #4}
			\item \textcolor{red}{exception: #5}
		\end{itemize}
}


%%%%%%%%%%%%%%%%%%%%%%%%%%%%%%%%%%%%%%%%%%%%%%%%%%%%%%%%%%%%%%%%%%%%%%%%%%%%%%%%%%%%%%
%	Exported Access Program Table
%		\accessProgramsTableStart
%			\row{one}{two}{three}{four}
%		\accessProgramsTableEnd
%%%%%%%%%%%%%%%%%%%%%%%%%%%%%%%%%%%%%%%%%%%%%%%%%%%%%%%%%%%%%%%%%%%%%%%%%%%%%%%%%%%%%%
\newcommand{\row}[4]{#1 & #2 & #3 & #4 ~\\ \hline}
\newcommand{\accessProgramsTableStart}{
\begin{tabular}{| l | l | l | l |}
\hline
\textbf{Routine name} & \textbf{In} & \textbf{Out} & \textbf{Exceptions}\\
\hline
}
\newcommand{\accessProgramsTableEnd}{
	\end{tabular}
}

% \newenvironment{\accessProgramsTable}{
% 	\begin{tabular}{| l | l | l | l |}
% 		\hline
% 		\textbf{Routine name} & \textbf{In} & \textbf{Out} & \textbf{Exceptions}\\
% 		\hline
% }{
% 	\end{tabular}
% }
%%%%%%%%%%%%%%%%%%%%%%%%%%%%%%%%%%%%%%%%%%%%%%%%%%%%%%%%%%%%%%%%%%%%%%%%%%%%%%%%%%%%%%


%%%%%%%%%%%%%%%%%%%%%%%%%%%%%%%%%%%%%%%%%%%%%%%%%%%%%%%%%%%%%%%%%%%%%%%%%%%%%%%%%%%%%%
%%%%%%%%%%%%%%%%%%%%%%%%%%%%%%%%%%%%%%%%%%%%%%%%%%%%%%%%%%%%%%%%%%%%%%%%%%%%%%%%%%%%%%
\begin{document}
%%%%%%%%%%%%%%%%%%%%%%%%%%%%%%%%%%%%%%%%%%%%%%%%%%%%%%%%%%%%%%%%%%%%%%%%%%%%%%%%%%%%%%
%%%%%%%%%%%%%%%%%%%%%%%%%%%%%%%%%%%%%%%%%%%%%%%%%%%%%%%%%%%%%%%%%%%%%%%%%%%%%%%%%%%%%%


%\maketitle
%
%\pagenumbering{roman}
%\tableofcontents
%\listoftables
%\listoffigures
%
%\begin{table}[bp]
%\caption{\bf Revision History}
%\begin{tabularx}{\textwidth}{p{3cm}p{2cm}X}
%\toprule {\bf Date} & {\bf Version} & {\bf Notes}\\
%\midrule
%Date 1 & 1.0 & Notes\\
%Date 2 & 1.1 & Notes\\
%\bottomrule
%\end{tabularx}
%\end{table}
%
%\newpage
%
%\pagenumbering{arabic}


%------------------------------------------------------------------------------
%------------------------------------------------------------------------------
%	Title Page
%------------------------------------------------------------------------------
%------------------------------------------------------------------------------


\begin{titlepage}

%------------------------------------------------------------------------------
%------------------------------------------------------------------------------
%	Document Information
%------------------------------------------------------------------------------
%------------------------------------------------------------------------------

% Team Info
\newcommand{\teamname}{REST Assured}
\newcommand{\teamnumber}{Team 31}

% Project Info
\newcommand{\course}{SE 3XA3}

% \docTitle defined in main tex doc
\title{\course: \docTitle \\ \teamname}

% Student 1
\newcommand{\studentAname}{Dawson Myers}
\newcommand{\studentAsid}{400005616}
\newcommand{\studentAmid}{myersd1}
\newcommand{\studentAInfo}{\studentAname \studentAsid \studentAmid}

% Student 2
\newcommand{\studentBname}{Yang Liu}
\newcommand{\studentBsid}{400038517}
\newcommand{\studentBmid}{liuy136}
\newcommand{\studentBInfo}{\studentBname \studentBsid \studentBmid}

% Student 3
\newcommand{\studentCname}{Brandon Roberts}
\newcommand{\studentCsid}{400018117}
\newcommand{\studentCmid}{roberb1}
\newcommand{\studentCInfo}{\studentCname \studentCsid \studentCmid}


\author{\teamnumber
	\\ \teamname
	\\ \studentAInfo
	\\ \studentBInfo
	\\ \studentCInfo
	%\\ Dawson Myers 400005616  mysersd1
	%\\ Yang Liu 400038517 liuy136
	%\\ Brandon Roberts 400018117 roberb1
}

\date{\dueDate}


%\maketitle

\centering
\parbox[t]{0.97\linewidth}{
	\centering \fontsize{20pt}{25pt}\selectfont % The first argument for fontsize is the font size of the text and the second is the line spacing

	\Large \course \\
	\huge \teamname \\
	\large\dueDate
	
	\vspace{0.1in}
	\hrule
	\vspace{0.1in}
	
	\Huge \docTitle
	
	\vspace{0.05in}
	\hrule
	\vspace{0.4in}
	
	\large \teamnumber
	
	\vspace{0.1in}
}

% Team member info
%-------------------------------------------------------------------------------
\begin{minipage}[c]{\linewidth}
%\fontsize{14pt}{25pt}\selectfont
			\large
			\centering
			\begin{tabular}{l c c}
				\studentAname & \studentAmid & \studentAsid \\
				\studentBname & \studentBmid & \studentBsid \\
				\studentCname & \studentCmid & \studentCsid \\
			\end{tabular}
\end{minipage}

%\vfill

%% Revision table
%%-------------------------------------------------------------------------------
%\begin{table}[bp]
%	\caption{\bf Revision History}
%	\begin{tabularx}{\textwidth}{p{3cm}p{2cm}X}
%		\toprule {\bf Date} & {\bf Version} & {\bf Notes}\\
%		\midrule
%		
%		06-Oct-2017 & 0.0 & Revision 0\\
%
%		\bottomrule
%	\end{tabularx}
%\end{table}

%\revisionTable

\end{titlepage}


%------------------------------------------------------------------------------
%------------------------------------------------------------------------------
%	Report
%------------------------------------------------------------------------------
%------------------------------------------------------------------------------

\pagenumbering{roman}
\def\thesection{\arabic{section}} 
\renewcommand\thesection{\arabic{section}} 
\renewcommand\thesubsection{\thesection.\arabic{subsection}}

\tableofcontents

\listoftables

\listoffigures


\newpage

\pagenumbering{arabic}
%==============================================================================
\section{Revision History}
%==============================================================================
\revisionTable


%------------------------------------------------------------------------------
% Module Heading
\section {RestStub Module}
% Module Type
% \subsection* {Template Module}
%------------------------------------------------------------------------------

\label{RestStub}

\newTemplateModule{RestStubT}
	%------------------------------------------------------
	{% Uses
	%------------------------------------------------------
		Constants
	}
	%------------------------------------------------------
	{% Exported Type
	%------------------------------------------------------
		RestStubT = ?
	}
	%------------------------------------------------------
	{% Exported Access Programs
	%------------------------------------------------------
		\accessProgramsTableStart
			%\row{name}{in}{out}{exception}
			\row{RestStub}{JSON}{}{}
			\row{toJSON}{}{JSON}{}
            \row{setRequestType}{String}{}{}
            \row{getRequestType}{}{}{}
            \row{\sout{runTest}}{Function}{}{}
		\accessProgramsTableEnd
	}
	%------------------------------------------------------
	{% State Variant
	%------------------------------------------------------
		label: String - User readable label of the test stub. \\
        identifier: String - Unique string used to identify test stubs. \\
        \textcolor{red}{requestType}: String - Type of request for the test: [GET, PUT, POST, DELETE] \\
        \textcolor{red}{requestData}: JSON - Data for the request. \\
        \textcolor{red}{responseData}: JSON - Response data for the request. \\
        \textcolor{red}{expectedData}: JSON - Expected response data. \\
        \textcolor{red}{resource: String - Resource location (HTTP)} \\
	}
	%------------------------------------------------------
	{% State Invariant
	%------------------------------------------------------
        none
	}
	%------------------------------------------------------
	{% Assumptions
	%------------------------------------------------------
		none
	}
	%------------------------------------------------------
	{% Access Routine Semantics
	%------------------------------------------------------
		%----------------
		\newAccessProgram{RestStub(data)}
			{% Input
				\textcolor{red}{JSON equivalent copy of an existing RestStub, or null.}
			}
			{% Transition
				The module constructor. Loads information from data if possible.
			}
			{% Output
				Assign fields to the data fields, if data is null the fields are set to defaults.
			}
			{% Exception
				none
			}
		%----------------
		\newAccessProgram{toJSON()}
			{% Input
				none
			}
			{% Transition
				none
			}
			{% Output
				\textcolor{red}{JSON equivalent of the object.}
			}
			{% Exception
				none
			}
		\newAccessProgram{setRequestType(rtype)}
			{% Input
				rtype
			}
			{% Transition
				Sets the value of requestType field to rtype
			}
			{% Output
				none
			}
			{% Exception
				none
			}
		\newAccessProgram{getRequestType()}
			{% Input
				none
			}
			{% Transition
				none
			}
			{% Output
				Value of requestType
			}
			{% Exception
				none
			}
			\oldAccessProgram{runTest(callback)}
			{% Input
				none
			}
			{% Transition
				Runs the test and executes the callback operation based on the test success.
			}
			{% Output
				none
			}
			{% Exception
				none
			}
	}
	%------------------------------------------------------
	{% Local Functions
	%------------------------------------------------------
		none
	}
	
\newpage


%------------------------------------------------------------------------------
% Module Heading
\section {RestChain Module}
% Module Type
% \subsection* {Template Module}
%------------------------------------------------------------------------------

\label{RestChain}

\newTemplateModule{RestChainT}
	%------------------------------------------------------
	{% Uses
	%------------------------------------------------------
		Constants
	}
	%------------------------------------------------------
	{% Exported Type
	%------------------------------------------------------
		RestChain = ?
	}
	%------------------------------------------------------
	{% Exported Access Programs
	%------------------------------------------------------
		\accessProgramsTableStart
			%\row{name}{in}{out}{exception}
			\row{RestChain}{JSON}{}{}
			\row{addTest}{String}{}{}
            \row{removeTest}{String}{}{MissingTestException}
            \row{moveTest}{}{String, int}{MissingTestException, IndexOutOfBounds}
            \row{\sout{runTests}}{\sout{Function}}{}{}
		\accessProgramsTableEnd
	}
	%------------------------------------------------------
	{% State Variant
	%------------------------------------------------------
		label: String - User readable label of the RestChain. \\
        identifier: String - Unique string used to identify RestChain. \\
        \textcolor{red}{reststublist}: List$<$String$>$ - List of RestChain identifiers to be run by the RestChain. \\
	}
	%------------------------------------------------------
	{% State Invariant
	%------------------------------------------------------
        none
	}
	%------------------------------------------------------
	{% Assumptions
	%------------------------------------------------------
		none
	}
	%------------------------------------------------------
	{% Access Routine Semantics
	%------------------------------------------------------
		\newAccessProgram{RestChain(data)}
			{% Input
				\textcolor{red}{JSON equivalent copy of an existing RestChain, or null.}
			}
			{% Transition
				The module constructor.
			}
			{% Output
				Assign fields to the data fields, if data is null the fields are set to defaults.
			}
			{% Exception
				none
			}
		\newAccessProgram{toJSON()}
			{% Input
				none
			}
			{% Transition
				none
			}
			{% Output
				\textcolor{red}{JSON equivalent of the object.}
			}
			{% Exception
				none
			}
			\newAccessProgram{addTest(testid)}
			{% Input
				String of the identifier for the RestStub to add to the RestChain.
			}
			{% Transition
				Appends the identifier, to the \textcolor{red}{reststublist.}
			}
			{% Output
				Json equivalent of the object.
			}
			{% Exception
				none
			}
		\newAccessProgram{removeTest(identifier)}
			{% Input
				Identifier of test to remove from the RestChain.
			}
			{% Transition
				Test identifier is removed from \textcolor{red}{reststublist.}
			}
			{% Output
				none
			}
			{% Exception
				MissingTestException: Thrown if the RestStub is not in \textcolor{red}{reststublist.}
			}
		\newAccessProgram{moveTest(identifier, newIndex)}
			{% Input
				Identifier of the RestStub, and new index in the list.
			}
			{% Transition
				Moves the position of the corresponding RestStub in \textcolor{red}{reststublist.}
			}
			{% Output
				none
			}
			{% Exception
				MissingTestException: Thrown if the RestStub is not in \textcolor{red}{reststublist.} 
				IndexOutOfBounds: newIndex is not a valid index for \textcolor{red}{reststublist.}
			}
		\oldAccessProgram{runTests(callback)}
			{% Input
				none
			}
			{% Transition
				Runs the tests in the RestChain and executes callback function with success boolean.
			}
			{% Output
				none
			}
			{% Exception
				none
			}
	}
	%------------------------------------------------------
	{% Local Functions
	%------------------------------------------------------
		none
	}
	
\newpage

%------------------------------------------------------------------------------
% Module Heading
\section {JsonComparer Module}
% Module Type
% \subsection* {Template Module}
%------------------------------------------------------------------------------

\label{JsonComparer}

\newModule{JsonComparer}
	%------------------------------------------------------
	{% Uses
	%------------------------------------------------------
		Constants
	}
	%------------------------------------------------------
	{% Exported Type
	%------------------------------------------------------
		none
	}
	%------------------------------------------------------
	{% Exported Access Programs
	%------------------------------------------------------
		\accessProgramsTableStart
			%\row{name}{in}{out}{exception}
			\row{compare}{JSON, JSON}{}{}
			\row{compareExact}{JSON, JSON}{JSON}{}
            \row{compareWithTolerance}{JSON, JSON, int}{}{}
		\accessProgramsTableEnd
	}
	%------------------------------------------------------
	{% State Variant
	%------------------------------------------------------
		none \\
	}
	%------------------------------------------------------
	{% State Invariant
	%------------------------------------------------------
        none
	}
	%------------------------------------------------------
	{% Assumptions
	%------------------------------------------------------
		none
	}
	%------------------------------------------------------
	{% Access Routine Semantics
	%------------------------------------------------------
		\newAccessProgram{compare(expected, received)}
			{% Input
				Two json objects to compare for similarities.
			}
			{% Transition
				none
			}
			{% Output
				True if received is a subset of expected.
			}
			{% Exception
				none
			}
		\newAccessProgram{compareExact(expected, received)}
			{% Input
				Two json objects to compare for similarities.
			}
			{% Transition
				none
			}
			{% Output
				Returns True if received is identical to expected. 
				\textcolor{red}{Otherwise, False is returned.}
			}
			{% Exception
				none
			}
		\newAccessProgram{compareWithTolerance(expected, received, mostDifferences)}
			{% Input
				Two json objects to compare for similarities, and an int for the maximum number of differences between the json objects.
			}
			{% Transition
				none
			}
			{% Output
				none
			}
			{% Exception
				none
			}
	}
	%------------------------------------------------------
	{% Local Functions
	%------------------------------------------------------
		none
	}
\newpage

%------------------------------------------------------------------------------
% Module Heading
\section {\textcolor{red}{JsonParser Module}}
% Module Type
% \subsection* {Template Module}
%------------------------------------------------------------------------------

\label{JsonParser}

\newModule{\textcolor{red}{JsonParser}}
	%------------------------------------------------------
	{% Uses
	%------------------------------------------------------
		Constants
	}
	%------------------------------------------------------
	{% Exported Type
	%------------------------------------------------------
		none
	}
	%------------------------------------------------------
	{% Exported Access Programs
	%------------------------------------------------------
		\accessProgramsTableStart
			%\row{name}{in}{out}{exception}
			\row{\textcolor{red}{prettify}}{JSON}{String}{}
			\row{\textcolor{red}{JSONtoParam}}{JSON}{String}{}
		\accessProgramsTableEnd
	}
	%------------------------------------------------------
	{% State Variant
	%------------------------------------------------------
		none \\
	}
	%------------------------------------------------------
	{% State Invariant
	%------------------------------------------------------
        none
	}
	%------------------------------------------------------
	{% Assumptions
	%------------------------------------------------------
		none
	}
	%------------------------------------------------------
	{% Access Routine Semantics
		\revAccessProgram{prettify(data)}
			{% Input
				json object to prettify (turn into user-readable Srting)
			}
			{% Transition
				none
			}
			{% Output
				Returns prettified version of data.
			}
			{% Exception
				none
			}
		\revAccessProgram{JSONtoParam(data)}
			{% Input
				json object to paramaterize.
			}
			{% Transition
				none
			}
			{% Output
				Returns a paramaterized version of data that can be used in HTTP requests.
			}
			{% Exception
				none
			}
	}
	%------------------------------------------------------
	{% Local Functions
	%------------------------------------------------------
		none
	}
	
\newpage
\newpage

%------------------------------------------------------------------------------
% Module Heading
\section {\textcolor{red}{JsonParser Module}}
% Module Type
% \subsection* {Template Module}
%------------------------------------------------------------------------------

\label{RestConsole}

\newModule{\textcolor{red}{RestConsole}}
	%------------------------------------------------------
	{% Uses
	%------------------------------------------------------
		none
	}
	%------------------------------------------------------
	{% Exported Type
	%------------------------------------------------------
		none
	}
	%------------------------------------------------------
	{% Exported Access Programs
	%------------------------------------------------------
		\accessProgramsTableStart
			%\row{name}{in}{out}{exception}
			\row{\textcolor{red}{restLog}}{String}{}{}
			\row{\textcolor{red}{reRenderConsole}}{}{}{}
		\accessProgramsTableEnd
	}
	%------------------------------------------------------
	{% State Variant
	%------------------------------------------------------
		none \\
	}
	%------------------------------------------------------
	{% State Invariant
	%------------------------------------------------------
        none
	}
	%------------------------------------------------------
	{% Assumptions
	%------------------------------------------------------
		none
	}
	%------------------------------------------------------
	{% Access Routine Semantics
		\revAccessProgram{restLog(message)}
			{% Input
				String message to add to the console.
			}
			{% Transition
				Adds the message to the console list. 
				Calls reRenderConsole to display the data to the user.
			}
			{% Output
				none
			}
			{% Exception
				none
			}
		\revAccessProgram{reRenderConsole()}
			{% Input
				none
			}
			{% Transition
				Displays the diagnostic console information to the user.
			}
			{% Output
				none
			}
			{% Exception
				none
			}
	}
	%------------------------------------------------------
	{% Local Functions
	%------------------------------------------------------
		none
	}
	

%------------------------------------------------------------------------------
% Module Heading
\section {RestSuite Module}
% Module Type
% \subsection* {Template Module}
%------------------------------------------------------------------------------

\label{RestSuite}

\newModule{RestSuite}
	%------------------------------------------------------
	{% Uses
	%------------------------------------------------------
		Constants, ProfileStore, IO, JsonParser, RestConsole
	}
	%------------------------------------------------------
	{% Exported Type
	%------------------------------------------------------
		none
	}
	%------------------------------------------------------
	{% Exported Access Programs
	%------------------------------------------------------
		\accessProgramsTableStart
			%\row{name}{in}{out}{exception}
			\row{display}{}{}{}
			\row{saveAllToDisk}{}{}{}
            \row{loadAllFromDisk}{}{}{}
            \row{saveSpecifiedToDisk}{String}{}{}
		\accessProgramsTableEnd
	}
	%------------------------------------------------------
	{% State Variant
	%------------------------------------------------------
        \textcolor{red}{currentRestChain: RestChain - Highlighted RestChain} \\
        \textcolor{red}{currentRestStub: RestStub - Highlighted RestStub} \\
        \textcolor{red}{currentProfileStore: ProfileStore - Highlighted ProfielStore} \\
	}
	%------------------------------------------------------
	{% State Invariant
	%------------------------------------------------------
        none
	}
	%------------------------------------------------------
	{% Assumptions
	%------------------------------------------------------
		none
	}
	%------------------------------------------------------
	{% Access Routine Semantics
	%------------------------------------------------------
		\newAccessProgram{display()}
			{% Input
				none
			}
			{% Transition
				none
			}
			{% Output
				Visual information for the Window to display to the user.
			}
			{% Exception
				none
			}
		\newAccessProgram{saveAllToDisk}
			{% Input
				none
			}
			{% Transition
				none
			}
			{% Output
				Downloads the application's state to the user's machine.
			}
			{% Exception
				none
			}
			\newAccessProgram{loadAllFromDisk()}
			{% Input
				none
			}
			{% Transition
				none
			}
			{% Output
				Updates the application's state with information from a selected file.
			}
			{% Exception
				none
			}
			%-------------------
			\newAccessProgram{saveSpecifiedToDisk(data)}
			{% Input
				none
			}
			{% Transition
				none
			}
			{% Output
				Saves specified information to a file.
			}
			{% Exception
				none
			}
			%-------------------
			\newAccessProgram{reRenderTestSelect()}
			{% Input
				none
			}
			{% Transition
				Renders the RestStub information to the UI.
			}
			{% Output
				none
			}
			{% Exception
				none
			}
			%-------------------
			\revAccessProgram{uiDeleteRestStub(identifier)}
			{% Input
				identifier for the RestStub to delete.
			}
			{% Transition
				Deletes the corresponding RestStub by performing deleteRestStub(identifier) on currentProfileStore.
			}
			{% Output
				none
			}
			{% Exception
				none
			}
			%-------------------
			\revAccessProgram{uiCopyRestStub(identifier)}
			{% Input
				identifier for the RestStub to copy.
			}
			{% Transition
				Copies the corresponding RestStub by performing copyRestStub(identifier) on currentProfileStore.
				Adds the new RestStub to currentRestChain by performing addTest with the new RestStub's identifier.
			}
			{% Output
				none
			}
			{% Exception
				none
			}
			%-------------------
			\revAccessProgram{uiCreateRestStub()}
			{% Input
			none
			}
			{% Transition
				Creates a new RestStub by performing createRestStub() on currentProfileStore.
				Adds the new RestStub to currentRestChain by performing addTest with the new RestStub's identifier.
			}
			{% Output
				none
			}
			{% Exception
				none
			}
			%-------------------
			\revAccessProgram{updateCurrentRestStub()}
			{% Input
			none
			}
			{% Transition
				Takes form data from the master detail form, and stores it into currentRestStub by using setters. Finally calls reRenderTestSelect().
			}
			{% Output
				none
			}
			{% Exception
				none
			}
			%-------------------
			\revAccessProgram{changeCurrentRestStub(identifier)}
			{% Input
				identifier for the RestStub to switch focus to.
			}
			{% Transition
				currentRestStub will point to the RestStub with the corresponding identifier after performing getRestStubFromID(identifier) on currentProfileStore.
			}
			{% Output
				none
			}
			{% Exception
				none
			}
			%-------------------
			\revAccessProgram{moveDownCurrentRestStub(identifier)}
			{% Input
				identifier for the RestStub to move down currentRestChain.
			}
			{% Transition
				The corresponding RestStub will be moved one position lower on the RestChain after performing moveRestStub (with appropriate parameters) on currentRestChain.
			}
			{% Output
				none
			}
			{% Exception
				none
			}
			%-------------------
			\revAccessProgram{displayCurrentRestStub()}
			{% Input
				none
			}
			{% Transition
				Takes information from currentRestStub, and displays it unto the master detail form.
			}
			{% Output
				none
			}
			{% Exception
				none
			}
			%-------------------
			\revAccessProgram{runAllTests()}
			{% Input
				none
			}
			{% Transition
				Performs all tests on RestStubs inside of currentRestChain.
				Sequentially calls runNextTest until all tests are run or until one test fails.
				If any RestStub fails a test, the UI notifies the user by calling failTests().
				If all RestStubs pass their tests, the UI notifies the user by calling succeedTests().
			}
			{% Output
				none
			}
			{% Exception
				none
			}
			%-------------------
			\revAccessProgram{runNextTest()}
			{% Input
				none
			}
			{% Transition
				Perform the next test on RestStubs inside of currentRestChain.
			}
			{% Output
				none
			}
			{% Exception
				none
			}
			%-------------------
			\revAccessProgram{failTests()}
			{% Input
				none
			}
			{% Transition
				Notifies the user that tests fails by displaying a message on the screen.
			}
			{% Output
				none
			}
			{% Exception
				none
			}
			%-------------------
			\revAccessProgram{succeedTests()}
			{% Input
				none
			}
			{% Transition
				Notifies the user that tests succeeded by displaying a message on the screen.
			}
			{% Output
				none
			}
			{% Exception
				none
			}
	}
	%------------------------------------------------------
	{% Local Functions
	%------------------------------------------------------
		none
	}
	
\newpage


%------------------------------------------------------------------------------
% Module Heading
\section {Constants Module}
% Module Type
% \subsection* {Template Module}
%------------------------------------------------------------------------------

\label{Constants}
%\newConstantsModule{module}{uses}{expConst}{expAccProg}{stateVar}{stateInvariant}
\newConstantsModule{Constants}
	%------------------------------------------------------
	{% Uses
	%------------------------------------------------------
		Constants
	}
	%------------------------------------------------------
	{% Exported Constants
	%------------------------------------------------------
		REQUEST\_TYPES: List$<$String$>$ - String of valid requests
		\textcolor{red}{ID\_LENGTH: int - Length of valid identifiers}
		\textcolor{red}{ID\_CHARACTERPOOL: String - String of valid characters for identifiers}
	}
	%------------------------------------------------------
	{% Exported Access Programs
	%------------------------------------------------------
	none
	}
	%------------------------------------------------------
	{% State Variant
	%------------------------------------------------------
		none\\
	}
	%------------------------------------------------------
	{% State Invariant
	%------------------------------------------------------
        none
	}
% 	%------------------------------------------------------
% 	{% Assumptions
% 	%------------------------------------------------------
% 		none
% 	}
% 	%------------------------------------------------------
% 	{% Access Routine Semantics
% 	%------------------------------------------------------
% 		%----------------
% 		
% 	}
% 	%------------------------------------------------------
% 	{% Local Functions
% 	%------------------------------------------------------
% 		none
% 	}
	
\newpage

%------------------------------------------------------------------------------
% Module Heading
\section {ProfileStore Module}
% Module Type
% \subsection* {Template Module}
%------------------------------------------------------------------------------

\label{ProfileStore}

\newModule{ProfileStore}
	%------------------------------------------------------
	{% Uses
	%------------------------------------------------------
		Constants
	}
	%------------------------------------------------------
	{% Exported Type
	%------------------------------------------------------
		
	}
	%------------------------------------------------------
	{% Exported Access Programs
	%------------------------------------------------------
		\accessProgramsTableStart
			%\row{name}{in}{out}{exception}
			\row{init}{}{}{}
			\row{getRestStubFromID}{String}{RestStub}{missingRestStub}
            \row{getRestChainFromID}{String}{RestChain}{missingRestChain}
            \row{newRestStub}{}{String}{}
            \row{newRestChain}{}{String}{}
            \row{copyRestStub}{String}{String}{missingRestStub}
            \row{copyRestChain}{String}{String}{missingRestChain}
            \row{generateIdentifier}{}{String}{}
            \row{newProfile}{}{String}{}
            \row{deleteRestStub}{String}{}{missingRestStub}
            \row{deleteRestChain}{String}{}{missingRestChain}
            \row{textcolor{red}{randomHash}}{String}{}{}
		\accessProgramsTableEnd
	}
	%------------------------------------------------------
	{% State Variant
	%------------------------------------------------------
		textcolor{red}{allRestStubs}: List$<$RestStub$>$ - List of all identifiers for RestStubs in memory. \\
        textcolor{red}{allRestChains}: List$<$RestChain$>$ - List of all identifiers for RestChains in memory. \\
        textcolor{red}{allIdentifiers}: List$<$String$>$ - List of all identifiers in memory. \\
        textcolor{red}{allProfiles}: List$<$Profile$>$ - List of all profiles in memory. \\
	}
	%------------------------------------------------------
	{% State Invariant
	%------------------------------------------------------
        none
	}
	%------------------------------------------------------
	{% Assumptions
	%------------------------------------------------------
		none
	}
	%------------------------------------------------------
	{% Access Routine Semantics
	%------------------------------------------------------
		\newAccessProgram{getRestStubFromID(identifier)}
			{% Input
				Ientifier of the RestChain to retrieve.
			}
			{% Transition
				none
			}
			{% Output
				RestChain with the identifier.
			}
			{% Exception
				none
			}
		\newAccessProgram{getRestChainFromID(identifier)}
			{% Input
				none
			}
			{% Transition
				none
			}
			{% Output
				none
			}
			{% Exception
				none
			}
		\newAccessProgram{newRestStub()}
			{% Input
				none
			}
			{% Transition
				Creates a new RestStub, and appends it to \textcolor{red}{allRestStubs}.
			}
			{% Output
				Identifier of the new RestStub.
			}
			{% Exception
				none
			}
		\newAccessProgram{copyRestStub(identifier)}
			{% Input
				Identifier of the RestStub to copy.
			}
			{% Transition
				Creates a deep copy of an existing RestStub, and appends it to \textcolor{red}{allRestStubs}.
			}
			{% Output
				Identifier of the new RestStub.
			}
			{% Exception
				none
			}
			\newAccessProgram{copyRestChain(identifier)}
			{% Input
				Identifier of the RestChain to copy.
			}
			{% Transition
				Creates a deep copy of an existing RestChain, and appends it to \textcolor{red}{allRestChains}.
			}
			{% Output
				New identifier.
			}
			{% Exception
				none
			}
		\newAccessProgram{newProfile()}
			{% Input
				none
			}
			{% Transition
				Creates a new Profile, and appends it to \textcolor{red}{allProfiles}.
			}
			{% Output
				Identifier of the new Profile.
			}
			{% Exception
				none
			}
		\newAccessProgram{deleteRestStub(identifier)}
			{% Input
				none
			}
			{% Transition
				Removes a RestStub from \textcolor{red}{allRestStubs}, and performs removeRestStub(identifier) on all Profile objects in \textcolor{red}{allProfiles}.
			}
			{% Output
				none
			}
			{% Exception
				none
			}
		\newAccessProgram{deleteRestChain(identifier)}
			{% Input
				none
			}
			{% Transition
				 Removes a RestChain from \textcolor{red}{allRestStubs}, and performs removeRestChain(identifier) on all Profile objects in \textcolor{red}{allRestChains}.
			}
			{% Output
				none
			}
			{% Exception
				none
			}
	}
	%------------------------------------------------------
	{% Local Functions
	%------------------------------------------------------
		none
	}
	
\newpage


%------------------------------------------------------------------------------
% Module Heading
\section {Profile Module}
% Module Type
% \subsection* {Template Module}
%------------------------------------------------------------------------------

\label{Profile}

\newModule{ProfileT}
	%------------------------------------------------------
	{% Uses
	%------------------------------------------------------
		RestStub, RestChain
	}
	%------------------------------------------------------
	{% Exported Type
	%------------------------------------------------------
		
	}
	%------------------------------------------------------
	{% Exported Access Programs
	%------------------------------------------------------
		\accessProgramsTableStart
			%\row{name}{in}{out}{exception}
			\row{init}{}{}{}
			\row{getRestStubs}{}{List$<$String$>$}{}
            \row{getRestChains}{List$<$String$>$}{}{}
            \row{addRestStub}{String}{}{}
            \row{addRestChain}{String}{}{}
            \row{removeRestStub}{String}{String}{}
            \row{removeRestChain}{String}{String}{}
		\accessProgramsTableEnd
	}
	%------------------------------------------------------
	{% State Variant
	%------------------------------------------------------
		loca\_RestStubs: List$<$String$>$ - List of identifiers for RestStubs in this Profile.\\
        local\_RestChains: List$<$String$>$ - List of identifiers for RestChains in this Profile.\\
        local\_identifiers: List$<$String$>$ -  List of identifiers in this Profile.\\
	}
	%------------------------------------------------------
	{% State Invariant
	%------------------------------------------------------
        none
	}
	%------------------------------------------------------
	{% Assumptions
	%------------------------------------------------------
		none
	}
	%------------------------------------------------------
	{% Access Routine Semantics
	%------------------------------------------------------
		\newAccessProgram{getRestStubs()}
			{% Input
				none
			}
			{% Transition
				none
			}
			{% Output
				copy of local\_RestStubs
			}
			{% Exception
				none
			}
		\newAccessProgram{getRestChains()}
			{% Input
				none
			}
			{% Transition
				none
			}
			{% Output
				Copy of local\_RestChains.
			}
			{% Exception
				none
			}
			\newAccessProgram{addRestStub()}
			{% Input
				Identifier of RestStub to add to Profile.
			}
			{% Transition
				Adds corresponding RestStub, and appends it to local\_RestStubs.
			}
			{% Output
				dentifier of the new RestStub.
			}
			{% Exception
				none
			}
		\newAccessProgram{addRestChain()}
			{% Input
				Identifier of RestChain to add to Profile.
			}
			{% Transition
				Creates a new RestChain, and appends it to local\_RestChains.
			}
			{% Output
				Identifier of the new RestChain.
			}
			{% Exception
				none
			}
		\newAccessProgram{removeRestStub(identifier)}
			{% Input
				Identifier of RestChain to remove from Profile.
			}
			{% Transition
				Removes the RestStub local\_RestStubs, if it exists.
				Performs the RemoveRestStub(identifier) on all TestChains in local\_RestChains.
			}
			{% Output
				none
			}
			{% Exception
				none
			}
		\newAccessProgram{removeRestChain(identifier)}
			{% Input
				Identifier of RestChain to remove from Profile.
			}
			{% Transition
				Removes the RestChain local\_RestChains, if it exists.
			}
			{% Output
				none
			}
			{% Exception
				none
			}
	}
	%------------------------------------------------------
	{% Local Functions
	%------------------------------------------------------
		none
	}
	
\newpage


%------------------------------------------------------------------------------
% Module Heading
\section {IO Module}
% Module Type
% \subsection* {Template Module}
%------------------------------------------------------------------------------

\label{IO}

\newModule{IO}
	%------------------------------------------------------
	{% Uses
	%------------------------------------------------------
		Constants
	}
	%------------------------------------------------------
	{% Exported Type
	%------------------------------------------------------
	}
	%------------------------------------------------------
	{% Exported Access Programs
	%------------------------------------------------------
		\accessProgramsTableStart
			%\row{name}{in}{out}{exception}
			\row{readFile}{String}{String}{MissingFileException, FileSizeError}
			\row{saveFile}{String, String}{JSON}{MissingFileException}
		\accessProgramsTableEnd
	}
	%------------------------------------------------------
	{% State Variant
	%------------------------------------------------------
		none
	}
	%------------------------------------------------------
	{% State Invariant
	%------------------------------------------------------
        none
	}
	%------------------------------------------------------
	{% Assumptions
	%------------------------------------------------------
		none
	}
	%------------------------------------------------------
	{% Access Routine Semantics
	%------------------------------------------------------
		\newAccessProgram{readFile(location)}
			{% Input
				Storage location of file to read.
			}
			{% Transition
				none
			}
			{% Output
				String of file contents.
			}
			{% Exception
				none
			}
		\newAccessProgram{saveFile(fileName, data)}
			{% Input
				Name of file to save to the user's computer, and the data to store in the file.
			}
			{% Transition
				none
			}
			{% Output
				String of file contents.
			}
			{% Exception
				none
			}
	}
	%------------------------------------------------------
	{% Local Functions
	%------------------------------------------------------
		none
	}
	
\newpage

%%%%%%%%%%%%%%%%%%%%%%%%%%%%%%%%%%%%%%%%%%%%%%%%%%%%%%%%%%%%%%%%%%%%%%%%%%%%%%%%
%%%%%%%%%%%%%%%%%%%%%%%%%%%%%%%%%%%%%%%%%%%%%%%%%%%%%%%%%%%%%%%%%%%%%%%%%%%%%%%%
%%%%%%%%%%%%%%%%%%%%%%%%%%%%%%%%%%%%%%%%%%%%%%%%%%%%%%%%%%%%%%%%%%%%%%%%%%%%%%%%

\end{document}


%%------------------------------------------------------------------------------
%% Module Heading
%\section* {Region Module}
%% Module Type
%\subsection* {Template Module}
%%------------------------------------------------------------------------------
%\newModule{ModuleName}
%	%------------------------------------------------------
%	{% Uses
%	%------------------------------------------------------
%		SomeOtherModule
%	}
%	%------------------------------------------------------
%	{% Exported Types
%	%------------------------------------------------------
%		Exported Types
%	}
%	%------------------------------------------------------
%	{% Exported Access Programs
%	%------------------------------------------------------
%		\accessProgramsTableStart
%		\row{name}{in}{out}{exception}
%		\accessProgramsTableEnd
%	}
%	%------------------------------------------------------
%	{% State Variant
%	%------------------------------------------------------
%		Variant
%	}
%	%------------------------------------------------------
%	{% State Invariant
%	%------------------------------------------------------
%		Invariant
%	}
%	%------------------------------------------------------
%	{% Assumptions
%	%------------------------------------------------------
%		No assumptions}
%	%------------------------------------------------------
%	{% Access Routine Semantics
%	%------------------------------------------------------
%		%----------------
%		\newAccessProgram{is\_validSegment($p_1$, $p_2$)}
%			{% Transition
%				$\mathit{none}$}
%			{% Output
%				
%			}
%			{% Exception
%				$\mathit{none}$
%			}
%		%----------------
%		\newAccessProgram{is\_validSegment($p_1$, $p_2$)}
%			{% Transition
%				$\mathit{none}$}
%			{% Output
%				
%			}
%			{% Exception
%				$\mathit{none}$
%			}		
%	}
%	%------------------------------------------------------
%	{% Local Functions
%	%------------------------------------------------------
%		%----------------
%		\newAccessProgram{is\_validSegment($p_1$, $p_2$)}
%			{% Transition
%				$\mathit{none}$}
%			{% Output
%				
%			}
%			{% Exception
%				$\mathit{none}$
%			}
%	}